\chapter[Préambule]{Préambule / SDL, OpenGL, GLEW et GLM}
Il m'a semblé nécessaire de faire un préambule, afin de prévenir de ce qui va suivre.
\\\\
Cet exercice a été réalisé sur une configuration différente, puisqu'il n'a pas été fait sur les ordinateurs du CREMI, pour la simple et bonne raison que c'étaient les vacances. Nous avons passé plus de 4 heures\footnote{Et il n'y a aucune exagération, vraiment, cet exercice a pris, au total, plus de 25 heures à être réalisé.} à essayer de faire fonctionner OpenGL sous Windows. Pourquoi avoir tenté sous Windows ? Une des forces d'OpenGL, c'est d'être compatible sous toutes les plateformes. Mais ces 4 heures ont suffi à nous faire abandonner l'idée. Nous avons réussi à faire fonctionner SDL, OpenGL, mais GLEW et GLM, c'était impossible (tout du moins, pas trivial). Nous sommes donc passés sous Linux.
\\\\
Mais alors que nous pensions que ce serait plus simple - l'installation l'était - il a fallu que les choses soient plus complexes également. Les fichiers sources, et corrections ne compilaient pas. Pourquoi ? Parce que la configuration du CREMI est entièrement personnalisée. Comment, nous n'en savions rien. Quelle solution avons-nous adoptée ? Nous avons tout refait depuis zéro. Nous avons appris à développer avec OpenGL en C++, puis nous avons tout traduit en C, au moins pour tout ce qui était possible. La seule chose que nous ne pouvions pas traduire était GLM, qui est pensée avec le C++. Heureusement, le compilateur étant g++, nous avons pu, ignoblement, insérer des instructions C++ dans nos fichiers sources C. Rien que de le dire, j'en ai envie de vomir. Mais nous n'avions aucun moyen d'outrepasser le namespace glm.
\\\\
Le résultat que vous avez ici peut donc ne pas compiler sur votre machine. Nous n'en savons rien. Nous mettrons donc des captures d'écran, au moins pour que le rapport se tienne à lui-même. Si vous parcourez le code, vous verrez également que le tout est bien plus massif que ce qui était fait en TP. Pourquoi, je n'en sais rien\footnote{A vrai dire, je pense savoir, puisque nous n'avons assisté à aucun des cours sur OpenGL...}. Mais nous avons tout de même, en quelques jours, appris l'OpenGL de zéro, tout en traduisant le tout en C. Et ça... On en tire une certaine fierté. Non pas d'un résultat parfait, ni correct, mais de savoir que nous nous sommes au moins approchés du résultat final, malgré tout.
\\\\
Le prochain chapitre - très court - traitera de l'utilisation du programme par l'utilisateur, et le suivant répondra aux questions soulevées par l'exercice.