Afin de pouvoir réaliser tout ceci, nous avons suivi un tutoriel pour apprendre rapidement à développer en OpenGL. Et comme nous sommes fainéants, nous avons suivi celui d'Open Classrooms, comme souvent, tout en gardant à l'esprit l'exercice demandé, afin de suivre la voie dont nous avions besoin.
\\\\
Nous sommes conscients, en comparant notre code aux corrections des TPs, que cette solution est immensément lourde. Nous nous sommes compliqués la tâche, mais aussi, nous avons du laisser tomber l'éclairage et les shaders. Les shaders utilisés ici sont ceux du tutoriel. Nous avons perdu beaucoup de temps à traduire du C++ vers le C, et ce, même partiellement, car nous n'avons pas pu passer toutes les limites (notamment pour GLM).
\\\\
L'exercice que nous rendons est donc incorrect, et incomplet, mais nous le rendons tout de même, car il contient une partie de ce qui était demandé, en OpenGL, ce qui, en soi, ne déroge pas vraiment aux consignes. Seule la méthode employée diffère.
\\\\
Dans le cas où le code ne compilerait pas - puisque nous avons une configuration probablement différente - nous avons mis en ligne quelques secondes du programme pour montrer le déplacement de la caméra, et les interactions des cubes. Il s'agit là de montrer au moins quelque chose si jamais le destin serait vraiment contre nous. Le framerate de la vidéo est grandement inférieur à celui du programme (qui tourne à 50 fps). C'est du au fait qu'il a fallu trouver une solution compatible avec Linux pour pouvoir enregistrer la fenêtre... Et c'est loin d'être parfait.
\\\\
Voici tout de même le lien :\\\href{https://youtu.be/pzBXBPMAAmg}{https://youtu.be/pzBXBPMAAmg}.
\\\\
Et si jamais besoin, pour une raison X ou Y, l'adresse du dépôt sur GitHub :\\\href{https://github.com/Pyraexyrin/SDL\_glSetRelativeKebabMode.git}{https://github.com/Pyraexyrin/SDL\_glSetRelativeKebabMode.git}.
\\\\
Cordialement.